This nifty document will guide you around the \hyperlink{namespaceHumDrum}{Hum\+Drum} library, describing both what each wing of the library does and what its sub-\/sections consist of.

\subsection*{Collections}

Collections is the wing of the \hyperlink{namespaceHumDrum}{Hum\+Drum} library that deals with any function performed on a series of (commonly) generic data.

\subsubsection*{Markov}

Markov is a subsection of collections that sets up the relevant data structure for working with Markov chains, a state-\/dependant prediction of future states.

\paragraph*{Markov\+State}

Represents one item of a markov chain. Holds some kind of state, a future value, and the probability that the future state will occur when its state is reached.

\paragraph*{Markov}

Markov is the actual class for working with these chains. Its constructor takes a sequence of data which is automatically turned into a list of Markov\+States. When a state is given to this object, it can either return the most likely future states or a state at random (a-\/la many message generators).

\subsubsection*{State\+Modifiers}

State\+Modifiers is a library suite for Hum\+Drum.\+Collections.\+Groups.\+State\+Object$<$\+T$>$ subclasses. These work as hodge-\/podgey implementations of state machines.

\paragraph*{Integer\+Counter}

Integer\+Counter is a pretty simple state machine. It takes a maximum value and a step-\/increase value (i.\+e, step increase value of 2 would make it go \{2, 4, 6\} etc.).

\subsubsection*{Groups}

Groups is a library for using a state machine (State\+Object$<$\+T$>$ defined in this class) to group information in a sequence.

\subsubsection*{Higher\+Order}

Higher\+Order is a library that mimics L\+I\+N\+Q-\/style queries for generic collection structures. These functions allow various other kinds of functions to be passed in as a parameter to control how these work.

\subsubsection*{Information}

Information is a very general library. Any function that pertains to extracting existing elements of a list as-\/is, or providing metadata about the list itself will wind up here. This library is the \char`\"{}other half\char`\"{} of the Transformations library, but it does not type of manipulation to the list.

\subsubsection*{Predicates}

Predicates is a library for dealing with analysis on sequences of boolean values.

\subsubsection*{Sections}

Sections is a library for parsing text. It defines various ways to group and extract strings from source material.

\subsubsection*{Transformations}

Transformations is another general collections library. The \char`\"{}other half\char`\"{} of Information, Transformations defines ways to manipulate a list to turn it into a different structure. This is either a reordered list or another data structure entirely.

\subsection*{Operations}

Operations is the wing of the \hyperlink{namespaceHumDrum}{Hum\+Drum} library pertaining to IO, or some very particular task.

\subsubsection*{Files}

Files is a subsection of Operations that relates to operations of files / directories.

\paragraph*{Directory\+Search}

Directory\+Search is a library for searching directories recursively for files meeting some type of criteria. Using method cascading, heavy refining of included files is made very easy with Directory\+Search.

\paragraph*{Line}

Line is a simple class that represents the line of a file. Lines are wrapped in this class so that their parent file and line number can be referenced even after it is separated from such a context.

\paragraph*{I\+Sequential\+Writer}

Sequential\+Writer is an interface defining two things\+: the writing of a file with an extension and the naming of such a file based on the directory. In short, anything that scans the directory to determine what the next file should be named can be a Sequential\+Writer

\paragraph*{Numerical\+Writer}

Numerical\+Writer is a Sequential\+Writer that scans the directory for numbered files, starting at 0.

\subsubsection*{Image\+Manager}

A library for calculating image metadata. This includes functions such as the average color, most similar image based on color, and image searching.

\subsubsection*{Logger}

Logger writes information to an Output\+Stream. Logger is commonly used on File\+Output\+Streams to create log files.

\subsubsection*{Servitor}

Servitor is a server program that listens on a given port and buffers input. Then, the client can read information from this buffer. This wraps the shockingly low-\/level .N\+ET network library.

\subsection*{Structures}

Structures is the wing of \hyperlink{namespaceHumDrum}{Hum\+Drum} that defines data structures.

\subsubsection*{Bindings\+Table}

Bindings\+Table contains functions for manipulating two lists of information, and manipulating lists of points. This can also be used a simple key-\/value database.

\subsubsection*{Direction}

A very simple enum for Up, Down, Left, Right.

\subsubsection*{Tree}

A simple implementation of a binary tree.

\subsection*{Traits}

Traits contains functions and classes relating to implementing trait-\/based programming in C\#. It allows you to use the custom definition of what a Class is so that classes which already satisfy some interface automatically qualify, and you can dynamically add methods to.

\subsubsection*{Class}

Class is a custom definition of the {\ttfamily class} keyword in C\#. It will scan the methods that a given class already has, adding them to its \char`\"{}method bank\char`\"{}. However, this also allows you to add named delegates, which will be used when scanning custom interfaces to see if a need is met.

\subsubsection*{Exception.\+cs}

Exceptions contains the exceptions that can possibly be thrown when working with traits. Many of these are empty, as there is no true workaround to the problems that may arise, and should only be caught for informational purposes.

\subsubsection*{\hyperlink{Implementor_8cs_source}{Implementor.\+cs}}

Implementor simply keeps track of what interfaces, both native and custom, a certain Class object implements.

\subsubsection*{Interface}

Interface is a custom version of a native {\ttfamily interface}. It, like the custom class, lets you dynamically add methods that you would like to be satiated for the Interface to be met.

\subsubsection*{Method}

Method is a custom definition of a Method. The closest native C\# idiom to method would be a {\ttfamily delegate}, although, Methods are meant to be named. Classes contain Method objects, which represent the methods that a class has, and custom-\/named delegates.

\subsubsection*{Trait}

Trait is the star of the Traits namespace. It allows you to set up a binding between a class and an interface, and ensure that all of the interface\textquotesingle{}s method requirements are met. If you\textquotesingle{}re going to be using Trait based programming, you\textquotesingle{}ll be accepting Trait objects, and instantiating them before you send them to your method. 